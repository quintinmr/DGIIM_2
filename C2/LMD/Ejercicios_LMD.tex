\documentclass[a4papper, 11pt]{article}
\usepackage{setspace}
\usepackage[utf8]{inputenc}
\usepackage[a4paper]{geometry}
\usepackage{amssymb}
\geometry{top=2.54cm, bottom=2.54cm, left=2.54cm, right=2.54cm}
\title{Ejercicios LMD}
\author{Quintín Mesa Romero }
\date{March 2022}

\begin{document}

\maketitle

\begin{flushleft}
	\textbf{Ejercicio 4.} Use el teorema de inducción para demostrar que: \newline
	
	\hfil $2^{n-1} \leq n!$ \newline
	
	Llamemos $P(n)$ a la desigualdad $2^{n-1} \leq n!$, donde $n$ es un número natural cualquiera. Razonemos por inducción sobre $n$ para probar que, efectivamente, la desigualdad es cierta $\forall n \in \omega$:
	
	\begin{itemize}
		\item $P(1)$ es cierta (\textbf{caso base}); en efecto:
		
		$2^{1-1} = 2^{0} = 1 = 1!$ $\Rightarrow P(1)$ es cierta.
		
		\item Sea $n$ un número natural tal que $1 \leq n$, y supongamos que $2^{n-1} \leq n!$, es decir, que $P(n)$ es cierta (\textbf{hipótesis de inducción}).
		
		\item En el \textbf{paso de inducción} el objetivo es demostrar que $2^{(n+1)-1} \leq (n+1)!$, es decir, que se verifica $P(n+1)$, lo cual es cierto. En efecto: 
		
		$P(n+1) = 2^{n} \leq (n+1)! \Leftrightarrow 2^{n} \leq (n+1)n!$
		
		$2^{n} \leq (n+1)n! \Leftrightarrow 2 \cdot 2^{n-1} \leq (n+1)n!$
		
		Como sabemos por la hipótesis de inducción que $2^{n-1} \leq n!$, el problema se reduce a probar que $2 \leq n+1 \Leftrightarrow 1 \leq n$. Lo cual es cierto, pues es el dominio de nuestra hipótesis. \newline
 
	\end{itemize}
	
	Por tanto, por el principio de inducción matemática, sabemos entonces que para todo número natural $n$ tal que $1 \leq n$ se cumple: \newline
	
	\hfil  $2^{n-1} \leq n!$ \newline
	
	\textbf{Ejercicio 5.} Demuestre mediante el teorema de inducción que: \newline
	
	\hfil $\prod_{i=1}^{n} \frac{2i-1}{2i} \leq \frac{1}{\sqrt{n+1}}$ \newline
	para todo $n \geq 1$ \newline
	
	\textbf{Caso base:} Veamos si la desigualdad se verifica para $n = 1$. \newline
	
	\hfil $\prod_{i=1}^{1} \frac{2i-1}{2i} \leq \frac{1}{\sqrt{2}} \Longleftrightarrow \frac{1}{2} \leq \frac{1}{\sqrt{2}}$ \newline
	
	\textbf{Hipótesis de inducción:} Suponemos que es cierto para $n$. Veamos si se verifica para $n+1$. \newline
	
	$\prod_{i=1}^{n+1} \frac{2i-1}{2i} \leq \frac{1}{\sqrt{n+2}} \Longleftrightarrow (\prod_{i=1}^{n} \frac{2i-1}{2i}) \cdot (\frac{2n + 1}{2n+2}) \leq \frac{1}{\sqrt{n+2}} \Longleftrightarrow \newline$
	
	$(\prod_{i=1}^{n} \frac{2i-1}{2i}) \cdot (\frac{2n + 1}{2n+2}) \leq \frac{1}{\sqrt{n+1}} \cdot \frac{2n + 1}{2n+2} \Longleftrightarrow \frac{1}{\sqrt{n+1}} \cdot \frac{2n + 1}{2n+2} \leq \frac{1}{\sqrt{n+2}} \Longleftrightarrow \newline$
	
	$\frac{2n + 1}{2n+2} \leq \frac{\sqrt{n+1}}{\sqrt{n+2}} \Longleftrightarrow \frac{4n^{2} + 4n + 1}{4n^{2} + 8n + 4} \leq \frac{n+1}{n+2} \Longleftrightarrow 4n^{3} + 12n^{2} + 9n + 2 \leq 4n^{3} + 12n^{2} + 12n + 4 \Longleftrightarrow \newline$
	
	$0 \leq 3n + 2$. Esto es cierto para todo $ n \in \omega$. \newline
	
	\textbf{Ejercicio 6.} Demuestre que todo número natural mayor que 1 puede ser expresado como producto de números primos. \newline
	
	Sea $P(i)$ el enunciado del tenor $P(i) = $ "el número natural $i$ puede expresarse como producto de números primos". Para probar su veracidad para todo $n \in \mathbb{N}$ aplicamos el segundo principio de inducción: Dado $n \in \mathbb{N}$, suponemos cierto $P(k)$ para todo $k \in \mathbb{N}$ tal que $1 < k < n$:\newline
	
	\textbf{Caso base:} 
	
	Distinguimos 2 casos: 
	\begin{itemize}
		\item Si $n$ es primo, $n = n-1$ $\Rightarrow$ cierto
		\item Si $n$ no es primo, se puede descomponer en producto de 2 factores $a,b \Rightarrow n = a \cdot b$ con $1 < a \leq b < n$
	\end{itemize}
	
	
	\textbf{Ejercicio 7.} Use el teorema de inducción para demostrar que $3^{n} + 7^{n} -2$ es divisible por $8$ para $n \ge 1$. \newline
	
	Sea $P(i)$ el enunciado $P(i) = $ "$3^{i} + 7^{i} -2$ es divisible por $8$". Para probar su veracidad para todo $n \in \mathbb{N}$ aplicamos el segundo principio de inducción: \newline
	
	\begin{itemize}
		\item \textbf{Caso base:} $n=1 \Rightarrow 3+7-2 = 8 | 8 \Rightarrow P(1)$ es cierta.
		\item \textbf{Hipótesis de inducción:} Dado $n \in \mathbb{N}$ suponemos que se cumple $P(k) \forall k \in \mathbb{N}$ tal que $1 \leq 1 \leq n$. Veamos que es cierto para $n+1$: \newline
		$3^{n+1} + 7^{n+1} -2 = (3^{n} + 7^{n} -2) (3+7) -3\cdot7^{n} - 7\cdot3^{n} + 18 = 10(3^{n} + 7^{n} -2) -21(7^{n+1} -2) -24 \Rightarrow 3^{n+1} + 7^{n+1} -2$ es múltiplo de $8$.
	\end{itemize}
	Por el segundo principio de inducción, el enunciado es cierto para todo $n \ge 1$. \newline
	
	\textbf{Ejercicio 18.} Sea $e$ la función dada por: \newline
	
	\hfil $e(a,0) = 1$,\newline
	
	\hfil     $e(a,b)=  \left\{ \begin{array}{lcc}
							  e(a^{2}, \frac{b}{2}) & \textit{si } b \textit{ es par} \\
							  \\  e(a^{2}, \frac{b-1}{2})a & \textit{si } b \textit{ es impar}
							\end{array}
						\right. $ \newline

    Demuestre por inducción que para cualesquiera números naturales $a$ y $b$, $e(a,b) = a^{b}$. Este método de "elevar" es, con justicia, muy afamado y se le ha llamado "Método de la Exponenciación Rápida..." muy usado en el ámbito de la criptografía moderna. \newline
    
    Fijemos un $a \in \omega$ arbitrario y hagamos la inducción sobre $b$. \newline
    
    Sea $P(b) = "e(a,b) = a^{b}$ $ \forall a,b \in \omega"$ \newline
    
    \textbf{Caso base:} $P(0)$ es cierta. En efecto: \newline
    $e(a,0) = 1 = a^{0}$ \newline
    
    \textbf{Hipótesis de inducción:} Supongamos ahora que $P(b)$ es cierto. Esto es, $b < b+1 \Rightarrow P(b+1)$ es cierta. \newline
    Hemos de distinguir dos casos: cuando $b+1$ es par y cuando $b+1$ es impar:
    
    \begin{itemize}
    	\item Para $b+1$ impar: \newline
    	$e(a,b+1) = e(a^{2, \frac{(b+1)-1}{2}})a = (a^{2})^{\frac{(b+1)-1}{2}} \cdot a = a^{b} \cdot a = a^{b+1}$
    	\item Para $b+1$ par: \newline
    	$e(a^{2}, \frac{b+1}{2}) = (a^{2})^{\frac{b+1}{2}} = a^{b+1}$
    \end{itemize}

Por tanto, por el Segundo Principio de Inducción Matemática, $P(b)$ se verifica para cualesquiera naturales $a$ y $b$.\newline



\textbf{Ejercicio 23.} Los números de Lucas son los números de la sucesión: \newline

\hfil     $L_{n}=  \left\{ \begin{array}{lcc}
2, & \textit{si } n = 0 \\
\\  1, & \textit{si }  n = 1 \\
\\ L_{n-1} + L_{n-2}, & \textit{si } 1 < n
\end{array}
\right. $ \newline

Demuestre que para todo número natural no nulo $n$ se cumple: \newline

\hfil $L_{n} < (\frac{7}{4})^{n}$ \newline

Sea el enunciado del tenor: $P(n) = $ "Si $n > 0$, entonces $L_{n} < (\frac{7}{4})^{n}$ ". 
Demostremos por el Segundo Principio de Inducción que $L_{n} < (\frac{7}{4})^{n}$ \newline

\textbf{Caso base:}  $n = 1  \Rightarrow L_{1} < \frac{7}{4} \Rightarrow 1 < \frac{7}{4}$ \newline 

\textbf{Hipótesis de inducción:} Sea $n \in \omega^{*}$, supongamos que $P(k)$ es cierta para $1 \leq k < n$ \newline

Para $n=2$:  \hfil $L_{2} = L_{0} + L_{2} = 2+1 < (\frac{7}{4})^{n}$ \newline

Para $n > 2$:
\hfil $L_{n} = L_{n-1} + L_{n-2} < (\frac{7}{4})^{n-1} + (\frac{7}{4})^{n-2} = (\frac{7}{4})^{n-2} \cdot (1+ \frac{7}{4}) = (\frac{7}{4})^{n-2} \cdot (\frac{11}{4})$ \newline
$< (\frac{7}{4}) ^{n-2} \cdot \frac{49}{16} = (\frac{7}{4})^{n-2} \cdot (\frac{7}{4})^{2} = (\frac{7}{4})^{n}$. \newline

Por el segundo principio de inducción, $P(n)$ es cierto $\forall n \ge 1$. \newline


\textbf{Ejercicio 25.} Demuestre que para todo número natural superior a $5 $, $n^{3} < n!$ \newline

\textbf{Caso base:} $6^{3} = 216 < 6! = 720$ \newline

\textbf{Hipótesis de inducción:} Supuesto para $n$ veamos que se cumple para $n+1.$ \newline
\hfil $(n+1)! = (n+1)n! > (n+1)n^{3} > (n+1)(n+1)^{2} = (n+1)^{3}$ \newline

*) $n^{3} > (n+1)^{2} = n^{2} + 1 + 2n < n^{2} + nn + nn = 3n^{2} < n^{3}$ \newline






	      
	      
	      
	
	
	
\end{flushleft}

\end{document}